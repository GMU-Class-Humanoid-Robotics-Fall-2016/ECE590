%%%%%%%%%%%%%%%%%%%%%%%%%%%%%%%%%%%%%%%%%
% Structured General Purpose Assignment
% LaTeX Template
%
% This template has been downloaded from:
% http://www.latextemplates.com
%
% Original author:
% Ted Pavlic (http://www.tedpavlic.com)
%
% Note:
% The \lipsum[#] commands throughout this template generate dummy text
% to fill the template out. These commands should all be removed when 
% writing assignment content.
%
%%%%%%%%%%%%%%%%%%%%%%%%%%%%%%%%%%%%%%%%%

%----------------------------------------------------------------------------------------
%	PACKAGES AND OTHER DOCUMENT CONFIGURATIONS
%----------------------------------------------------------------------------------------

\documentclass{article}

\usepackage{fancyhdr} % Required for custom headers
\usepackage{lastpage} % Required to determine the last page for the footer
\usepackage{extramarks} % Required for headers and footers
\usepackage{graphicx} % Required to insert images
\usepackage{lipsum} % Used for inserting dummy 'Lorem ipsum' text into the template
\usepackage{amsmath}
% Margins
\topmargin=-0.45in
\evensidemargin=0in
\oddsidemargin=0in
\textwidth=6.5in
\textheight=9.0in
\headsep=0.25in 

\linespread{1.1} % Line spacing

% Set up the header and footer
\pagestyle{fancy}
\lhead{\hmwkAuthorName} % Top left header
\chead{\hmwkClass\ (\hmwkClassInstructor\ \hmwkClassTime): \hmwkTitle} % Top center header
\rhead{\firstxmark} % Top right header
\lfoot{\lastxmark} % Bottom left footer
\cfoot{} % Bottom center footer
\rfoot{Page\ \thepage\ of\ \pageref{LastPage}} % Bottom right footer
\renewcommand\headrulewidth{0.4pt} % Size of the header rule
\renewcommand\footrulewidth{0.4pt} % Size of the footer rule

\setlength\parindent{0pt} % Removes all indentation from paragraphs

%----------------------------------------------------------------------------------------
%	DOCUMENT STRUCTURE COMMANDS
%	Skip this unless you know what you're doing
%----------------------------------------------------------------------------------------

% Header and footer for when a page split occurs within a problem environment
\newcommand{\enterProblemHeader}[1]{
\nobreak\extramarks{#1}{#1 continued on next page\ldots}\nobreak
\nobreak\extramarks{#1 (continued)}{#1 continued on next page\ldots}\nobreak
}

% Header and footer for when a page split occurs between problem environments
\newcommand{\exitProblemHeader}[1]{
\nobreak\extramarks{#1 (continued)}{#1 continued on next page\ldots}\nobreak
\nobreak\extramarks{#1}{}\nobreak
}

\setcounter{secnumdepth}{0} % Removes default section numbers
\newcounter{homeworkProblemCounter} % Creates a counter to keep track of the number of problems

\newcommand{\homeworkProblemName}{}
\newenvironment{homeworkProblem}[1][Problem \arabic{homeworkProblemCounter}]{ % Makes a new environment called homeworkProblem which takes 1 argument (custom name) but the default is "Problem #"
\stepcounter{homeworkProblemCounter} % Increase counter for number of problems
\renewcommand{\homeworkProblemName}{#1} % Assign \homeworkProblemName the name of the problem
\section{\homeworkProblemName} % Make a section in the document with the custom problem count
\enterProblemHeader{\homeworkProblemName} % Header and footer within the environment
}{
\exitProblemHeader{\homeworkProblemName} % Header and footer after the environment
}

\newcommand{\problemAnswer}[1]{ % Defines the problem answer command with the content as the only argument
\noindent\framebox[\columnwidth][c]{\begin{minipage}{0.98\columnwidth}#1\end{minipage}} % Makes the box around the problem answer and puts the content inside
}

\newcommand{\homeworkSectionName}{}
\newenvironment{homeworkSection}[1]{ % New environment for sections within homework problems, takes 1 argument - the name of the section
\renewcommand{\homeworkSectionName}{#1} % Assign \homeworkSectionName to the name of the section from the environment argument
\subsection{\homeworkSectionName} % Make a subsection with the custom name of the subsection
\enterProblemHeader{\homeworkProblemName\ [\homeworkSectionName]} % Header and footer within the environment
}{
\enterProblemHeader{\homeworkProblemName} % Header and footer after the environment
}
   
%----------------------------------------------------------------------------------------
%	NAME AND CLASS SECTION
%----------------------------------------------------------------------------------------

\newcommand{\hmwkTitle}{Assignment\ \#4} % Assignment title
\newcommand{\hmwkDueDate}{Tuesday,\ September\ 27,\ 2016} % Due date
\newcommand{\hmwkClass}{ECE\ 590} % Course/class
\newcommand{\hmwkClassTime}{19:20} % Class/lecture time
\newcommand{\hmwkClassInstructor}{Lofaro} % Teacher/lecturer
\newcommand{\hmwkAuthorName}{Corbin Wilhelmi} % Your name

%----------------------------------------------------------------------------------------
%	TITLE PAGE
%----------------------------------------------------------------------------------------

\title{
\vspace{2in}
\textmd{\textbf{\hmwkClass:\ \hmwkTitle}}\\
\normalsize\vspace{0.1in}\small{Due\ on\ \hmwkDueDate}\\
\vspace{0.1in}\large{\textit{\hmwkClassInstructor\ \hmwkClassTime}}
\vspace{3in}
}

\author{\textbf{\hmwkAuthorName}}
\date{} % Insert date here if you want it to appear below your name

%----------------------------------------------------------------------------------------

\begin{document}

\maketitle

%----------------------------------------------------------------------------------------
%	TABLE OF CONTENTS
%----------------------------------------------------------------------------------------

%\setcounter{tocdepth}{1} % Uncomment this line if you don't want subsections listed in the ToC

\newpage
\tableofcontents
\newpage

%----------------------------------------------------------------------------------------
%	PROBLEM 1
%----------------------------------------------------------------------------------------

% To have just one problem per page, simply put a \clearpage after each problem

\begin{homeworkProblem}
Make the Hubo (not the DRC-Hubo) squat so its waist is 0.5m off the ground, then make it stand up straight so it is 0.8m off the ground.  Have it repeat this 4 times.  Use IK methods to complete assignment.


\problemAnswer{ % Answer
Using the notation and examples from "Introduction To Robotics Mechanics and control" by John Craig:

\begin{align*}
^{0}_{1}T &= \begin{bmatrix}
c\theta_1 & -s\theta_1 & 0. & l_1 \\
s\theta_1 & c\theta_1 & 0. & 0. \\
0. & 0. & 1. & 0. \\
0. & 0. & 0. & 1.
\end{bmatrix}\\
^{1}_{2}T &= \begin{bmatrix}
c\theta_2 & -s\theta_2 & 0. & l_2 \\
s\theta_2 & c\theta_2 & 0. & 0. \\
0. & 0. & 1. & 0. \\
0. & 0. & 0. & 1.
\end{bmatrix}\\
^{2}_{3}T &= \begin{bmatrix}
c\theta_3 & -s\theta_3 & 0. & l_3 \\
s\theta_3 & c\theta_3 & 0. & 0. \\
0. & 0. & 1. & 0. \\
0. & 0. & 0. & 1.
\end{bmatrix}
\end{align*}
If we are looking for the hip joint of the Hubo robot then we are looking for the $[x \text{ } y \text{ } z]^T$ of the center of mass above the hip.  Solving for the full forward kinematics reveals: 

\begin{align*}
^{0}_{3}T &= \begin{bmatrix}
c\theta_1 & -s\theta_1 & 0. & l_1 \\
s\theta_1 & c\theta_1 & 0. & 0. \\
0. & 0. & 1. & 0. \\
0. & 0. & 0. & 1.
\end{bmatrix} 
\begin{bmatrix}
c\theta_2 & -s\theta_2 & 0. & l_2 \\
s\theta_2 & c\theta_2 & 0. & 0. \\
0. & 0. & 1. & 0. \\
0. & 0. & 0. & 1.
\end{bmatrix} 
\begin{bmatrix}
c\theta_3 & -s\theta_3 & 0. & l_3 \\
s\theta_3 & c\theta_3 & 0. & 0. \\
0. & 0. & 1. & 0. \\
0. & 0. & 0. & 1.
\end{bmatrix}
\end{align*} 

From class we know that the following is true:

\begin{align*}
\theta_1 &= \cos^{-1}\frac{x^2 + y^2 - d_1^2 -d_2^2}{2d_1d_2} \\
\theta_2 &= \tan^{-1}\left(\frac{y(d_1 + d_2\cos(\theta_2)) - x(d_2\sin(\theta_2)}{x(d_1 + d_2\cos(\theta_2)) - y(d_2\sin(\theta_2)} \right)
\end{align*}


Since the top part of the robot (above the hip) must be at $y=0$ (in robot coordinate system if $x$ is positive forward and rotation is around $z$) for the duration of the movement, it is then inferred that:

\begin{align*}
0 &= \theta_1 + \theta_2 + \theta_3 \\
\theta_3 &= -\theta_1 - \theta_2
\end{align*}

By plugging in the x and y desired values $\theta_1$, $\theta_2$, $\theta_3$ were solved. 

}
\end{homeworkProblem}

%----------------------------------------------------------------------------------------
%	PROBLEM 2
%----------------------------------------------------------------------------------------
------------------------------------------------------------------------

\end{document}
